\section{Desarrollo}

\subsection{Programa}

El programa que hicimos permite aplicar ciertos operadores de derivación a una imagen especificada por línea de comandos. Los operadores implementados son Roberts, Prewitt,  Sobel y Frei-Chen. El primero tiene matrices de 2x2 y detecta bordes en diagonal. Los otros tres tienen matrices de 3x3 y detectan bordes verticales u horizontales. Se pueden invocar tanto las versiones de propósito general para los primeros tres, como las versiones de los operadores que utilizan instrucciones SIMD. 

Para cada operador solicitado el programa aplica la matriz correspondiente en X, luego en Y, y finalmente suma los resultados. Para el caso de Sobel, también permite aplicar convolución solamente en X o solamente en Y.

Utilizamos la biblioteca \textbf{OpenCV} para manejar las imágenes convenientemente, es decir, dejamos la carga, el guardado de imágenes y el ``aplanado'' a escala de grises a dicha biblioteca. El grupo se limitó a crear una interfaz que contenga lo mencionado y a implementar los algoritmos de los operadores de detección de bordes en lenguaje ensamblador. 




%%%%%%%%%%%%%%%%%%%
% Acá en vez de contar todo el chamuyo de como se deriva si de derecha a izquierda o si es mejor 
% saturar de una forma u otra, creo que sería más productivo y fructífero mencionar los detalles de implementación
% de los operadores ya que son todos distintos y ya que la magia está en utilizar bien simd para los distintos casos
% but feel free to edit me - JM.
%%%%%%%%%%%%%%%%%%%%%

\subsection{Implementación}

%esto serían observaciones generales
% Respecto del procesamiento de datos en paralelo vía-SIMD lo que se hizo fue procesar los pixeles de las imágenes de la siguiente forma. Por un lado sabemos que la imagen tiene un ancho múltiplo de 16 pixeles lo cual facilita el \emph{fetch} de los mismos a los registros \texttt{xmm}. Además tenemos en cuenta que la imagen es de 8 bits en escala de grises, es decir, cada pixel es un número de 1 byte que según la intensidad oscila en valores enteros entre 0 y 255. 
% 
% Cada registro \texttt{xmm} tiene 128 bits de ancho, o lo que es igual, 16 bytes. Como el ancho de la imagen tiene las características mencionadas anteriormente, esto nos permite visualizar a la imagen en algo de $n$ columnas de 16 bytes de ancho, justamente el ancho de los registros y $n$ es un número entero cosa que nos permite en $n$ viajes a memoria cargar toda una fila en los registros y poder operar con ella convenientemente.
% 
% Esto es muy interesante porque antes se hacían $k$ accesos a memoria (donde $k$ es el ancho en pixeles de la imagen). Ahora se traen en ráfaga $\frac{k}{16}$ pixeles (si la imagen está en escala de grises). Esto minimiza notablemente la penalización que se paga por acceder a la memoria RAM que como sabemos es mucho más lenta en términos de velocidad que los registros internos del procesador.
% 
% Estas observaciones valen en caso que la operatoria se haga con valores enteros. Para el caso de Frei-Chen que tiene un $\sqrt{2}$ en la matriz esto deja de ser cierto ya que en ese caso, se opera directamente con números reales representados con precisión simple de 32-bits.

\subsubsection{Precisión numérica} %%esto creo que es general, de última yo lo uso en roberts y lo pongo ahí sino.

Respecto de la forma de operar aritméticamente, surge un inconveniente si tenemos 16 pixeles (almacenados como bytes) cuando tenemos que hacer una resta por ejemplo, ya que podríamos estar perdiendo información al saturar. Por ejemplo:

\begin{center}
\begin{minipage}{0.30 \textwidth}
$\begin{pmatrix}
a_{11} & a_{12} \\
a_{21} & a_{22}  \\
\end{pmatrix}$
\end{minipage}
\ \ 
 \begin{minipage}{0.30 \textwidth}
$\begin{pmatrix}
15 & 255 \\
1 & -255 \\
\end{pmatrix}$
\end{minipage}
\end{center}

En este caso si la convolución se hace al $a_{11}$ con la matriz de Roberts quedaría: $a_{11} = a_{22} - a_{21}$ que en nuestro caso se resume a...

Entonces para que en las restas de números no se pierda información en el medio lo que se hace es lo siguiente: traemos los bytes pero luego los desempaquetamos como \emph{words} de forma tal de no perder información con las restas. Luego los empaquetamos saturados para grabarlos en la imagen destino convenientemente. 

\pagebreak
Gráficamente sería algo como: 

% \includegraphics[scale=0.55]{/home/connor/Escritorio/fetch.pdf}

Sin embargo, estas observaciones valen en caso que la operatoria se haga con valores enteros. Para el caso de Frei-Chen que tiene un $\sqrt{2}$ en la matriz esto deja de ser cierto ya que en ese caso, se opera directamente con números reales representados con precisión simple de 32-bits.

\subsubsection{Operador de Roberts}

El operador de Roberts presenta las siguientes matrices de convolución para $x$ e $y$ respectivamente.



\begin{center}
\begin{minipage}{0.30 \textwidth}
$\begin{pmatrix}
1 & 0 \\
0 & -1  \\
\end{pmatrix}$
\end{minipage}
\ \ 
 \begin{minipage}{0.30 \textwidth}
$\begin{pmatrix}
0 & 1 \\
-1 & 0 \\
\end{pmatrix}$
\end{minipage}
\end{center}

Dada la precondición que dice que el ancho de las imágenes es múltiplo de 16 bytes y que cada pixel está representado con 1 byte, entonces, podemos visualizar a la imagen de forma tal que esta tiene $n$ columnas de 16 bytes de ancho. Además al usar los registros \texttt{xmm} de 16 bytes de tamaño, podemos efectivamente para una fila de tamaño $k$ pixeles hacer nada más que $\frac{k}{16}$ accesos a memoria para traer toda la fila en cuestión cuando antes no quedaba otra opción que traer todos los pixeles de uno en uno. 

El algoritmo que se encarga de realizar el operador de Roberts está dividido en dos partes independientemente de la dirección en la que se está pasando el operador. 

En primer lugar, como el ancho de la imagen es múltiplo de 16 bytes, se optó por dividir dicho ancho en múltiplos de 16 bytes, que como se mencionó antes es el tamaño de los registros \texttt{xmm}. Entonces se tienen por lo menos $n$ columnas de 16 bytes de ancho y se procede trabajar con las $n-1$ primeras de la siguiente forma: 
\begin{itemize}
\item Máscara X: Se cargan los 16 bytes de la fila actual en un registro y luego se cargan los 16 bytes de la fila siguiente pero desplazados uno. Es decir, si la fila actual es $x$ y la siguiente es $x'$ entonces se carga en un registro el $x_i, x_{i+1}, x_{i+2}, ... , x_{i+15}$ y en el otro $x'_{i+1}, x'_{i+2}, x'_{i+3}, ... , x'_{i+16}$ donde $i$ es el número de la columna de la imagen. 
\item Máscara Y: Este caso es análogo al anterior solo que el desplazamiento se hace en la fila actual, osea, se carga el $x_{i+1}, ..., x_{i+16}$ en un registro y en el siguiente el $x'_i, ... x'_{i+15}$
\end{itemize}

En ambos casos la idea del algoritmo es mantener las cosas lo más simple posible, de forma tal de que en este punto lo único que se hace es restar cada uno de los 16 bytes (que están separados en grupos de 8 words) y luego reempaquetarlos y dejarlos en la imagen destino. 

Luego se procesa la última columna de la siguiente manera:
\begin{itemize}
\item Mascara X: Se carga la fila actual y la siguiente sin desplazamiento pero a la fila siguiente se la \emph{shiftea} a izquierda un byte de forma tal de tener en un registro $x_i, x_{i+1}, x_{i+2}, ... , x_{i+15}$ para la fila actual y en otro para la siguiente: $x'_{i+1}, x'_{i+2}, x'_{i+3}, ..., x'_{i+15}, 0$. De esta manera logramos cargar la fila como corresponde (sin pasarnos del límite que tenemos que tener) y operarla correctamente luego de desplazarla dentro del registro. 
\item Máscara Y: Este caso es prácticamente igual al anterior. Se cargan sin desplazamiento tanto la fila actual como la siguiente. Luego se desplaza dentro del registro la fila actual con un \emph{shift} y se pone un 0 en la última columna de la fila siguiente de forma tal de dejar el borde en 0 mediante una máscara de bits dado que ese último pixel no es procesable.
\end{itemize}

En ambos casos se realizan las operaciones aritméticas pertinentes y luego se guarda el resultado donde corresponde. 

El volcado de datos a memoria se realiza solamente cuando ambas máscaras han sido pasadas. Antes se utiliza un registro \texttt{xmm} como acumulador para evitar un acceso que sería innecesario.



%acá se pueden poner imágenes... opcional igual. En caso negativo explico un poco mejor sino se entiende...

%\includegraphics[scale=0.60]{/home/connor/Escritorio/dibujo.pdf}



\subsubsection{Operador de Prewitt}

Matrices de convolución para el operador de Prewitt en $x$ e $y$ respectivamente.

\begin{center}
\begin{minipage}{0.30 \textwidth}
$\begin{pmatrix}
-1 & 0 & 1 \\
-1 & 0 & 1 \\
-1 & 0 & 1 \\
\end{pmatrix}$
\end{minipage}
\ \ 
 \begin{minipage}{0.30 \textwidth}
$\begin{pmatrix}
-1 & -1 & -1 \\
0 & 0 & 0  \\
1 & 1 & 1 \\
\end{pmatrix}$
\end{minipage}
\end{center}

Como podemos ver, en el caso del operador de Prewitt consta, tanto para X como para Y, de matrices de $3x3$ toda de números naturales, así que podemos trabajar con operaciones sobre enteros. Utilizamos, para realizar este operación, los registros registros simd, para poder procesar en paralelo. Intentamos ir procesando de la máxima cantidad de pixeles posibles, así que terminamos realizando un algoritmo que llega a procesar de a 14 pixeles en paralelo, el cual se detalla a continuación.

Como la matriz de convolución es de $3x3$ y el valor a calcular es el del centro de la misma, lo primero que notamos es que todos los bordes de la imagen no se van a poder calcular, así que lo que se hace es poner 'a mano' un 0 en sus respectivas posiciones. Otro punto a tener en cuenta es que, como se comentó anteriormente, a pesar de poder levantar de a 16 bytes, las operaciones se realizan sobre word, así que en principio vamos a tener el doble de registros ocupados que la cantidad que levantamos.

La idea del algoritmo en si es, principalmente, recorrer toda la imagen fuente e ir guardando los resultados en la imagen destino. Para esto cabe destacar que la primer fila entera, se saltea y se pone todo en 0(negro) y lo mismo hacemos con la ultima fila. En cuanto el ciclo que se realiza, lo que hacemos es tener un bucle anidado, que simulan el recorrido de de la imagen en sobre cada fila y, para cada fila, sobre las columna.
Dentro del bucle de las columnas, se aplica la convolución sobre x y sobre y, como se explica a continuación.

\begin{itemize}
\item Mascara X: Para esta mascara, lo que hacemos es en en donde estamos parado, que es el valor centrar del primer pixel a calcular, tomamos la los 16 pixeles de la fila de arriba, la actual y la de abajo, desplazados 1 a la izquierda, ya que ese dato es necesario para calcular los pixeles. Una vez que se tienen estas 3 lineas cargadas de 16 bytes cada una, se pasan a 6 lineas de 8 words cada una, para así poder hacer los cálculos sin perder precisión. De esta forma, y  para reducir significativamente los accesos a memoria, utilizamos 6 de los 8 registros xmm para guardar datos, con lo cual debemos adaptar el algoritmo para que se puedan hacer las cuentas con dos registros. Esta mascara esta dividida en dos partes
  \begin{itemize}
    \item Los 8 primeros pixeles: para poder calcular estos pixeles lo que hicimos es, utilizando los tres registros xmm que contienen estos pixeles, primero hacemos las tres restas, que esto es restar los tres registros así como están (ie. $res=-a1-a2-a3$) y luego, para poder realizar las sumas, despasamos cada registro en dos posiciones, poniendo donde estaba el primer pixel, el tercero, el segundo el cuarto, etc.
    De esta manera, quedan las ultimas dos posiciones libres, las cuales se completan con los dos primeros pixeles de la parte alta. Una vez que se tiene esto, se suman al resultando anterior. 
    \item Los 8 últimos pixeles: de estos pixeles los que vamos a calcular son las primeras 6 posiciones, para esto primero hacemos la resta igual con en el item anterior, y después lo que hacemos es desplazar cada registro dos pixeles, para poder hacer la suma correspondiente. Si es como obtenemos 6 valores calculados. 
  \end{itemize}
Por ultimo lo que hacemos es poner todos estos pixeles un un solo registro, saturando a byte sin signo, y por ultimo moverlos a la posición correspondiente en la imagen destino.
\item Máscara Y: Para esta mascara, solamente necesitamos la linea de arriba y la de abajo del pixel actual. Si que cargamos las dos lineas y dividimos en 4 de 8 words cada una. Nuevamente dividimos esta parte en dos
  \begin{itemize}
    \item Los 8 primeros pixeles: Lo que hacemos acá es sumar y restar los registros correspondientes, 3 veces pero en cada paso vamos desplazando cada registro para sacar el pixel en la primer posición, y poniendo en la ultima posición el primer pixel del registro que contiene a los 8 posteriores.
    \item Los 8 últimos pixeles: en este paso se realiza los mimos pasos que en los primeros 8, con la salvedad que no se rellena en cada desplazamiento, ya que solo necesitamos los primeros 6 pixeles.
  \end{itemize}
  Una vez hecho esto, se procede a unir los resultados, saturando sin signo para que queden valores entre 0 y 255. Para terminar, lo que se hacer es levantar los valores guardaos con la mascara x, sumarlos al actual y volverlos a guardar.
\end{itemize}
Ya habiendo calculado estos pixeles, lo que se hace es avanzar el iterador de la columna unas 14 posiciones, e iterar hasta que nos pasemos del ancho de la imagen.

\subsubsection{Operador de Sobel}

Matrices de Sobel en $x$ e $y$

\begin{center}
\begin{minipage}{0.30 \textwidth}
$\begin{pmatrix}
-1 & 0 & 1 \\
-2 & 0 & 2 \\
-1 & 0 & 1 \\
\end{pmatrix}$
\end{minipage}
\ \ 
 \begin{minipage}{0.30 \textwidth}
$\begin{pmatrix}
-1 & -2 & -1 \\
0 & 0 & 0  \\
1 & 2 & 1 \\
\end{pmatrix}$
\end{minipage}
\end{center}

En esta parte no vamos a detallar mucho, ya que es muy similar a la forma de calcular Prewitt, las únicas diferencias a destacar son, que para poder ver si se debe hacer en realizar la mascara de X o la de Y, se va preguntando, basado en los parámetros XOrder e YOrder, si hay que entrar en ese calculo, o si hay que saltarlo. Y por otro lado, la otra modificación es que se suma o resta dos veces (en lugar de una) en los lugares correspondiente para que cumpla con la matriz de convolución correspondiente. Salvando estas diferencias, el método es casi igual al de Prewitt y es por eso que no se entra en detalle.

\subsubsection{Operador de Frei-Chen}

Frei-Chen en $x$ e $y$.

\begin{center}
\begin{minipage}{0.30 \textwidth}
$\begin{pmatrix}
-1 & 0 & 1 \\
-\sqrt{2} & 0 & \sqrt{2} \\
-1 & 0 & 1 \\
\end{pmatrix}$
\end{minipage}
\ \ 
 \begin{minipage}{0.30 \textwidth}
$\begin{pmatrix}
-1 & -\sqrt{2} & -1 \\
0 & 0 & 0  \\
1 & \sqrt{2} & 1 \\
\end{pmatrix}$
\end{minipage}
\end{center}