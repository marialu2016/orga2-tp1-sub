\section{Apéndice A: Manual de usuario}

El programa realizado permite aplicar diferentes implementaciones de detección de bordes a imágenes. Tanto el nombre de las imagen fuente como los operadores se indican por línea de comandos de la siguiente manera (suponiendo que el ejecutable está en ``exe/bordes''):


   {\small \verb=$ exe/bordes {fuente} {destino} {operadores} =}

    
Donde:

%esto es una lista
    * \texttt{fuente}: es la imagen de origen; puede tener cualquier formato soportado por la función cvLoadImage de OpenCV; los formatos más comunes están soportados;
    
    * \texttt{destino}: es el nombre de las imágenes de salida (generadas por el programa) SIN INCLUIR LA EXTENSIÓN; los archivos generados tendrán ese nombre más un sufijo que indica el operador utilizado, más la extensión; la extensión, el formato y el tamaño son obtenidos de la imagen de entrada;
    
    * \texttt{operadores}: una o más claves separadas por espacio; cada clave representa una implementación particular de detección de bordes; las claves posibles y sus significados son los que siguen:

{\small
\begin{verbatim}    
                                                                      
    USO:    ./bordes $src $dest ${lista de operadores}                
                                                                      
    - $src: nombre del archivo de origen                              
    - $dest: nombre del archivo de salida SIN EXTENSIÓN               
    - ${lista de operadores}: una o más de las siguientes claves:     
                                                                      
             CLAVE | OPERADOR  | DIRECCION | IMPLEMENTACIÓN           
            -------+-----------+-----------+----------------          
              r1   |  Roberts  |    XY     |  asm + sse               
              r2   |  Prewitt  |    XY     |      "                  
              r3   |   Sobel   |    X      |      "                  
              r4   |   Sobel   |    Y      |      "                  
              r5   |   Sobel   |    XY     |      "                  
              r6   | Frei-chen |    XY     |      "                  
                   |           |           |                          
              a1   |  Roberts  |    XY     |asm g. purpose            
              a2   |  Prewitt  |    XY     |      "                  
              a3   |   Sobel   |    X      |      "                  
              a4   |   Sobel   |    Y      |      "                  
              a5   |   Sobel   |    XY     |      "                  
                   |           |           |                          
              cv3  |   Sobel   |    X      |    OpenCV                
              cv4  |   Sobel   |    Y      |      "                  
              cv5  |   Sobel   |    XY     |      "                  
                   |           |           |                          
            c1...c6|   (idem)  |   (idem)  |      C                   
                   |           |           |                          
              byn  |   Escala de grises    |                          
                   |           |           |                          
     El programa generará una imagen por cada operador solicitado. El 
 nombre de los archivos de salida será $dest más un guión bajo más el 
 operador utilizado más la extensión (obtenida del archivo fuente).   
\end{verbatim}  }                                        

Por ejemplo, si tenemos el ejecutable en \verb="exe/bordes"= y una imagen en \verb="pics/lena.bmp"=, llamando al programa como

{\small \verb=  $ exe/bordes pics/lena.bmp pics/lena byn r6 =}

Se aplicará el operador de Frei-Chen en SIMD y se generará el archivo \verb="pics/lena_r6.bmp" =. Además el programa mostrará por salida estándar la cantidad aproximada de clocks de procesador insumida por cada implementación.

    