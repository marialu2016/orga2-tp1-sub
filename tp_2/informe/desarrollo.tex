\section{Desarrollo}

\subsection{El programa}

El programa realizado es en realidad una extensión del presentado anteriormente.
Permite aplicar los mismos operadores de derivación que aquella versión: de
\textbf{Roberts}, \textbf{Prewitt} y \textbf{Sobel} realizados con la arquitectura
de propósito general y la versión de \emph{OpenCV} del operador de \textbf{Sobel},
además de las nuevas implementaciones agregadas de \textbf{Roberts}, \textbf{Prewitt},
\textbf{Sobel} y \textbf{Frei-chen} hechas con la tecnología \emph{SIMD}.

Para cada operador solicitado el programa aplica la matriz correspondiente en X,
luego en Y, y finalmente suma los resultados. La imagen resultante es grabada y
se muestra por pantalla la cantidad aproximada de clocks de procesador insumida.
Para el caso de Sobel, también permite aplicar convolución solamente en X o
solamente en Y, tanto en la implementacióne básica como en la optimizada.

Utilizamos la biblioteca \textbf{OpenCV} para manejar las imágenes convenientemente,
es decir, dejamos la carga, el guardado de imágenes y el ``aplanado'' a escala
de grises a dicha biblioteca.


\subsection{Implementación}

Dado que este trabajo consiste en una optimización de un algoritmo ya implementado,
cuando comenzamos a hacerlo teníamos bastante claro \emph{qué} teníamos que hacer.
Lo que quedaba por averiguar era \emph{cómo} hacerlo; nos referimos por supuesto al
diseño de los algoritmos de cómputo en paralelo.

Tal es así que la tarea en un principio parecía sencilla: esta vez no nos trabaríamos
con todas las incertidumbres y errores surgidos en el trabajo anterior, sólo tendríamos
que reproducir los mismos algoritmos levantando los datos de a ``paquetes'' y
realizando sobre ellos más o menos las mismas operaciones que antes.

Poco a poco se fue haciendo evidente, sin embargo, que esto no era cierto; la
programación con el modelo SIMD requiere un enfoque completamente distinto y 
tiene sus propias complicaciones. Al desarrollar un algoritmo de este tipo, se 
presentan muchas posibilidades de diseño, limitaciones y además requiere un
trabajo adicional para acomodar los datos: empaquetarlos, desempaquetarlos,
mezclarlos, etc...

El \emph{qué}, aquello que conocíamos, era solamente el algoritimo a alto nivel.
Lo que había que hacer era recorrer la imagen fuente y, por cada pixel, hallar el
``producto interno'' del entorno del punto con la matriz de convolución en X. Es
decir, calcular la suma de productos elemento a elemento de la matriz contra una
submatriz del dibujo del mismo tamaño. Ese resultado daría positivo o negativo
dependiendo de la ``dirección'' del borde y de la matriz usada; inmediatamente
lo saturaríamos a 0 quedándonos solamente con los bordes en una dirección y
sentido.

Luego haríamos exactamente lo mismo con la matriz de convolución en Y. Una vez
hecho esto tenemos dos valores no negativos que representan bordes en dos direcciones
perpendiculares; entonces los sumamos y, saturando esa suma a 255 la escribiríamos 
en la imagen destino como un pixel en la misma posición en la que estábamos parado
en la imagen fuente.


\subsubsection{Programación con SIMD}

En esta segunda parte asumimos que las imágenes tienen un ancho múltiplo de 16.
Esto implica, por un lado, que ya no es necesario calcular el \emph{widthstep}
como hacíamos antes. Como \emph{OpenCV} deja cada fila de pixeles alineada a 4
bytes en memoria, teníamos que calcular el mínimo múltiplo de 4 mayor o igual al
ancho de la imagen. Ahora en cambio el width (ancho) y el widthstep coinciden.

Además esto nos evita accesos a memoria, pues en nuestras anteriores implementaciones
	guardábamos el widthstep en un registro para movernos por los mapas de bits pero
accedíamos a memoria para consultar el width para saber cuándo terminar el bucle.

El hecho de que el ancho sea múltiplo de 16 también puede aprovecharse para operar
con los registros \textbf{XMM}, donde caben justamente 16 pixeles (cada uno de un byte).
Por ejemplo podemos dividir cada fila en bloques contiguos de 16 pixeles, levantando
cada bloque de una vez hacia un registro \textbf{XMM}.

Esto es justamente lo que hicimos en el caso de \textbf{Roberts} y en el de
\textbf{Frei-chen} (en este caso lo hicimos de a 8). Pero, como veremos más adelante,
no es necesario basarse en esa condición: las implementaciones de \textbf{Prewitt} y
\textbf{Sobel} barren cada fila de a 14 lugares, y están pensados para imágenes 
de cualquier ancho.

Para ello usamos la instrucción \textbf{\texttt{MOVQ}} en el caso de Frei-chen
(para levantar cada bloque de 8 pixeles en la parte baja de un registro \textbf{XMM}) y la
instrucción \textbf{\texttt{MOVDQU}} en los demás casos (para levantar bloques de
16 pixeles en registros \textbf{XMM}).

Una observación importante es que para operar con esos bytes sin signo (valores de
0 a 255) necesitamos extenderles el signo. Si operáramos usando bytes no podríamos
representar valores negativos ni mayores a 255, lo cual es necesario para aplicar
las matrices de convolución.

Para ello nos valimos de las instrucciones \textbf{\texttt{PUNPCKHBW}} y 
\textbf{\texttt{PUNPCKLBW}} que permiten desempaquetar la parte alta y baja
respectivamente de un \textbf{XMM} transformando los 8 bytes en 16 words (completando con
ceros). Esto se hace usando como operando fuente un registro con ceros.

Luego cada implementación realiza con esas words el procesamiento necesario,
explicado en las próximas secciones, tratándolas como enteros \textbf{signados}.
La excepción a esto es \textbf{Frei-chen}, que debe volver a desempaquetar los datos
en dwords y castearlos a flotantes de precisión sencilla. El casteo se realiza con
la instrucción \textbf{\texttt{CVTDQ2PS}}.

Luego del procesamiento correspondiente a la aplicación de las matrices, llegamos
siempre a tener los valores de las derivadas en registros \textbf{XMM} como 8 words signadas.
Para poder volcar los valores necesitamos saturarlos a bytes sin signo (a 0 y a 255)
y además empaquetarlos de a 16 bytes en registros \textbf{XMM}. Todo esto se logra con la 
instrucción \textbf{\texttt{PACKUSWB}}, que justamente toma dos registros \textbf{XMM} con
8 words signadas cada uno y las empaqueta en un solo registro como bytes sin signo
con saturación. Solo falta aplicar \textbf{\texttt{MOVDQU}} o \textbf{\texttt{MOVQ}}
nuevamente para escribir los valores definitivos en el destino.


\subsubsection{Operador de Roberts}

El operador de Roberts presenta las siguientes matrices de convolución para $x$ e $y$ respectivamente.

\begin{center}
\begin{minipage}{0.30 \textwidth}
$\begin{pmatrix}
1 & 0 \\
0 & -1  \\
\end{pmatrix}$
\end{minipage}
\ \ 
 \begin{minipage}{0.30 \textwidth}
$\begin{pmatrix}
0 & 1 \\
-1 & 0 \\
\end{pmatrix}$
\end{minipage}
\end{center}

Este operador se diferencia de los demás por el hecho de tener matrices de 2x2,
con sólo dos valores significativos (no ceros). Con registros de propósito general
esto no parecía marcar una gran diferencia, pues las operaciones se hacían
secuencialmente y no nos afectaba mucho la cantidad que hubiera que efectuar.

Pero ahora que estamos implementándolo con paralelización, este operador resulta
más simple que los demás.

En primer lugar, como el ancho de la imagen es múltiplo de 16, se optó por cargar
los datos en columnas de a 16, que como se mencionó antes es el tamaño de los registros \texttt{xmm}. Entonces se tienen $n$ columnas de 16 bytes de ancho y se procede trabajar con las $n-1$ primeras de la siguiente forma: 

\begin{itemize}
\item Máscara X: Se cargan los 16 bytes de la fila actual en un registro y luego se cargan los 16 bytes de la fila siguiente pero desplazados uno. Es decir, si la fila actual es $x$ y la siguiente es $x'$ entonces se carga en un registro el $x_i, x_{i+1}, x_{i+2}, ... , x_{i+15}$ y en el otro $x'_{i+1}, x'_{i+2}, x'_{i+3}, ... , x'_{i+16}$ donde $i$ es el número de la columna de la imagen. 
\item Máscara Y: Este caso es análogo al anterior solo que el desplazamiento se hace en la fila actual, osea, se carga el $x_{i+1}, ..., x_{i+16}$ en un registro y en el siguiente el $x'_i, ... x'_{i+15}$
\end{itemize}

En ambos casos la idea del algoritmo es mantener las cosas lo más simple posible, de forma tal de que en este punto lo único que se hace es restar cada uno de los 16 bytes (que están separados en grupos de 8 words) y luego reempaquetarlos y dejarlos en la imagen destino. 

Luego se procesa la última columna de la siguiente manera:
\begin{itemize}
\item Mascara X: Se carga la fila actual y la siguiente sin desplazamiento pero a la fila siguiente se la \emph{shiftea} a izquierda un byte de forma tal de tener en un registro $x_i, x_{i+1}, x_{i+2}, ... , x_{i+15}$ para la fila actual y en otro para la siguiente: $x'_{i+1}, x'_{i+2}, x'_{i+3}, ..., x'_{i+15}, 0$. De esta manera logramos cargar la fila como corresponde (sin pasarnos del límite que tenemos que tener) y operarla correctamente luego de desplazarla dentro del registro. 
\item Máscara Y: Este caso es prácticamente igual al anterior. Se cargan sin desplazamiento tanto la fila actual como la siguiente. Luego se desplaza dentro del registro la fila actual con un \emph{shift} y se pone un 0 en la última columna de la fila siguiente de forma tal de dejar el borde en 0 mediante una máscara de bits dado que ese último pixel no es procesable.
\end{itemize}

En ambos casos se realizan las operaciones aritméticas pertinentes y luego se guarda el resultado donde corresponde. 

El volcado de datos a memoria se realiza solamente cuando ambas máscaras han sido pasadas. Antes se utiliza un registro \textbf{XMM} como acumulador para evitar un acceso que sería innecesario.


\subsubsection{Operadores de Prewitt y de Sobel}

Matrices de Prewitt:

\begin{center}
\begin{minipage}{0.30 \textwidth}
$\begin{pmatrix}
-1 & 0 & 1 \\
-1 & 0 & 1 \\
-1 & 0 & 1 \\
\end{pmatrix}$
\end{minipage}
\ \ 
 \begin{minipage}{0.30 \textwidth}
$\begin{pmatrix}
-1 & -1 & -1 \\
0 & 0 & 0  \\
1 & 1 & 1 \\
\end{pmatrix}$
\end{minipage}
\end{center}

Matrices de Sobel:

\begin{center}
\begin{minipage}{0.30 \textwidth}
$\begin{pmatrix}
-1 & 0 & 1 \\
-2 & 0 & 2 \\
-1 & 0 & 1 \\
\end{pmatrix}$
\end{minipage}
\ \ 
 \begin{minipage}{0.30 \textwidth}
$\begin{pmatrix}
-1 & -2 & -1 \\
0 & 0 & 0  \\
1 & 2 & 1 \\
\end{pmatrix}$
\end{minipage}
\end{center}

La implementaciones de estos operadores son análogas entre sí, por eso las 
describimos juntas aquí.

Como podemos ver estos operadores tienen matrices de $3x3$ de números enteros.
Esto presenta un panorama más complejo que el de Roberts. En aquel caso, bastaba
con levantar dos filas de 16 pixeles (desplazadas una posición entre sí) para hallar
el resultado de una fila completa también de 16 pixeles.

Ahora la situación es distinta, pues 3 filas de 16 pixeles alcanzan para calcular
los valores de los 14 pixeles centrales. En un principio se nos ocurrieron dos
opciones: una sería levantar bloques más grandes (de 18) tomando un pixel adicional
en cada costado, y acomodarlos adecuadamente en los registros \textbf{XMM} para
poder calcular el resultado no de 14 sino de 16 pixeles por iteración. Así vamos
a poder completar de a 16 pixeles en la imagen destino.

Sin embargo optamos en este caso por la segunda opción: levantar los datos de a 16
y movernos de a 14. Esto obviamente no tiene en cuenta la restricción de que el 
ancho es múltiplo de 16. Por lo tanto cuando avancemos de a 14 es posible que en algún
momento nos ``pasemos de largo''.

Al principio creímos que las últimas columnas (menos de 14) tenían que ser calculadas
fuera del ciclo, porque si intentamos acceder a memoria más allá de los límites de las
imágenes podríamos estar escribiendo basura en algunos lugares del resultado e incluso
en lugares de la memoria que no nos corresponden.
 
Pero luego nos dimos cuenta de que no había tantos problemas: donde termina cada fila,
si nos pasamos escribiendo simplemente vamos a empezar a escribir datos en la fila
siguiente (que luego pisaremos con la información válida cuando la hayamos calculado);
por último, para la última fila que escribimos tampoco hay problema, pues recordemos
que la última fila escrita es la anteúltima del dibujo.

Para esto cabe destacar que la primer fila entera, se saltea y se pone todo en 0(negro)
y lo mismo hacemos con la ultima fila. Para cada las demás filas se realiza lo siguiente:

\begin{itemize}
\item Mascara X: Para esta mascara, lo que hacemos es en en donde estamos parado, que es el valor centrar del primer pixel a calcular, tomamos la los 16 pixeles de la fila de arriba, la actual y la de abajo, desplazados 1 a la izquierda, ya que ese dato es necesario para calcular los pixeles. Una vez que se tienen estas 3 lineas cargadas de 16 bytes cada una, se pasan a 6 lineas de 8 words cada una, para así poder hacer los cálculos sin perder precisión. De esta forma, y  para reducir significativamente los accesos a memoria, utilizamos 6 de los 8 registros xmm para guardar datos, con lo cual debemos adaptar el algoritmo para que se puedan hacer las cuentas con dos registros. Esta máscara está dividida en dos partes
  \begin{itemize}
    \item Los 8 primeros pixeles: para poder calcular estos pixeles lo que hicimos es, utilizando los tres registros xmm que contienen estos pixeles, primero hacemos las tres restas, que esto es restar los tres registros así como están (ie. $res=-a1-a2-a3$) y luego, para poder realizar las sumas, desplazamos cada registro en dos posiciones, poniendo donde estaba el primer pixel, el tercero, el segundo el cuarto, etc.
    De esta manera, quedan las ultimas dos posiciones libres, las cuales se completan con los dos primeros pixeles de la parte alta. Una vez que se tiene esto, se suman al resultando anterior. 
    \item Los 8 últimos pixeles: de estos pixeles los que vamos a calcular son las primeras 6 posiciones, para esto primero hacemos la resta igual con en el item anterior, y después lo que hacemos es desplazar cada registro dos pixeles, para poder hacer la suma correspondiente. Si es como obtenemos 6 valores calculados. 
  \end{itemize}
Por ultimo lo que hacemos es poner todos estos pixeles un un solo registro, saturando a byte sin signo, y por ultimo moverlos a la posición correspondiente en la imagen destino.
\item Máscara Y: Para esta mascara, solamente necesitamos la linea de arriba y la de abajo del pixel actual. Si que cargamos las dos lineas y dividimos en 4 de 8 words cada una. Nuevamente dividimos esta parte en dos
  \begin{itemize}
    \item Los 8 primeros pixeles: Lo que hacemos acá es sumar y restar los registros correspondientes, 3 veces pero en cada paso vamos desplazando cada registro para sacar el pixel en la primer posición, y poniendo en la ultima posición el primer pixel del registro que contiene a los 8 posteriores.
    \item Los 8 últimos pixeles: en este paso se realiza los mimos pasos que en los primeros 8, con la salvedad que no se rellena en cada desplazamiento, ya que solo necesitamos los primeros 6 pixeles.
  \end{itemize}
  Una vez hecho esto, se procede a unir los resultados, saturando sin signo para que queden valores entre 0 y 255. Para terminar, lo que se hacer es levantar los valores guardados con la mascara x, sumarlos al actual y volverlos a guardar.
\end{itemize}
Ya habiendo calculado estos pixeles, lo que se hace es avanzar el iterador de la columna unas 14 posiciones, e iterar hasta que nos pasemos del ancho de la imagen.

Como se ve, una vez que tenemos la derivada en X la escribimos en el destino como
si ya hubiéramos terminado, y una vez que tenemos la derivada en Y levantamos
aquellos valores y los sumamos. En el caso de Sobel, que permite aplicar convolución
solamente en X y solamente en Y, se realiza sólo uno de los pasos.

Si bien parecería más eficiente guardarse las derivadas en X no en memoria sino en
algún XMM hasta obtener las derivadas en Y, a la hora de hacer esas implementaciones
no nos quedó ningún registro disponible a tal fin de manera que resolvimos dejarlo
como se explicó.

\subsubsection{Operador de Frei-Chen}

Frei-Chen en $x$ e $y$.

\begin{center}
\begin{minipage}{0.30 \textwidth}
$\begin{pmatrix}
-1 & 0 & 1 \\
-\sqrt{2} & 0 & \sqrt{2} \\
-1 & 0 & 1 \\
\end{pmatrix}$
\end{minipage}
q\ \ 
 \begin{minipage}{0.30 \textwidth}
$\begin{pmatrix}
-1 & -\sqrt{2} & -1 \\
0 & 0 & 0  \\
1 & \sqrt{2} & 1 \\
\end{pmatrix}$
\end{minipage}
\end{center}

Frei-chen es el operador que se agregó en esta segunda parte del trabajo. Además
es el único que requiere procesamiento en punto flotante pues algunos de los
coeficientes de sus matrices son números reales (raíces de dos).
Con los registros de propósito general este procesamiento habría sido un tanto
complicado.

Usando los registros XMM (de 128 bits) como cuatro valores de punto flotante de 
32 bits (precisión simple) pudimos hacer una implementación con procesamiento 
paralelo de forma tan simple como lo hicimos con números enteros. La única 
diferencia importante es que en este caso entran 4 valores por registro XMM (con
enteros usamos 8 valores de 16 bits).

Las matrices de \textbf{Frei-chen} son muy parecidas a las otras de 3x3 que 
usamos (\textbf{Prewitt} y \textbf{Sobel}), de manera que los algoritmos pueden
pensarse de una forma parecida pese a la diferencia en la cantidad de valores
empaquetados.

Siguiendo el mismo esquema podríamos cargar tres filas de 8 pixeles (en lugar de
16) usando dos registros XMM para cada fila. Con los valores de esos 3x8 pixeles
podríamos calcular los valores de 6 pixeles (en lugar de 14), que son los 6
centrales de la fila del medio.

Sin embargo en este punto tomamos otro enfoque, para poder recorrer las
imágenes realmente de a 8 y no de a 6. La idea es la siguiente: numeremos las 
columnas desde 0 y supongamos que en la primera iteración levantamos desde la
columna 1 hasta la 8 (en lugar desde la 0 hasta la 7 como sería más típico).
Estos datos nos van a permitir calcular las derivadas para 6 pixeles, desde la 
columna 2 hasta la 7. El pixel de la columna 8 no lo podemos calcular pues para
ello necesitamos información de las columnas 7, 8 y 9 (nos falta la 9) así como
tampoco podemos calcular el de la columna 9 (pues nos faltan los datos de la fila
9 y la 10).

Lo que pensamos fue, en esa iteración (además de escribir los datos de los 6 pixeles
resueltos) guardarnos de alguna manera la información correspondiente a las dos últimas
columnas en los registros \textbf{XMM}. Entonces, en el siguiente paso levantaremos
(moviéndonos de a 8 posiciones) las columnas 9-16, pero además tendremos información
de las columnas 7 y 8 (del paso anterior). En definitiva, vamos a poder resolver 8
puntos, aquellos dos que nos habían quedado ``colgados'' (el 8 y el 9) más los 6
pixeles centrales del bloque levantado (10-15).

Usando este mecanismo es claro que podemos efectivamente avanzar de a 8 filas. Ahora
bien, ¿cómo nos guardamos la información de esas columnas? Usamos un mecanismo al
que llamamos \emph{comprimir columnas}. Mirando la matriz en X, observamos que 
podemos operar verticalmente, calculando paralelamente para cada columna el valor de la primera
fila más raíz de dos por la segunda más la tercera. Como se ve, estos son los coeficientes
por los que necesitamos multiplicar, más allá del signo.

En el código fuente y en adelelante nos referimos como $(n)$ a la columna $n$ comprimida.
Al resultado buscado, es decir a la derivada en la columna $i$, sea en X o en Y,
lo llamamos $[i]$. Una vez que las columnas están comprimidas de esta manera, la derivada
de un pixel se halla como una resta de columnas comprimidas.

$ [i] = (i+1) - (i) $

Asíque acomodando los datos de manera adecuada resulta sencillo hallar las derivadas en X
a partir de las columnas así comprimidas. Para el caso de Y, la compresión por columnas
es distinta: en este caso ni siquiera interesa el valor de la fila del medio; el valor
comprimido es simplemente la tercera fila menos la primera.

Es decir que nos quedamos con la diferencia entre las filas sin importar cuál vamos
a multiplicar luego por raíz de dos. El cálculo de cada derivada se reduce a...

$ [i] = (i-1) + \sqrt{2}(i) + (i+1) $

La idea de todo esto es que de una iteración a la otra queremos guardarnos los
valores comprimidos de las dos últimas columnas, tanto en X como en Y. Para esto
nos alcanza un único registro \textbf{XMM} (\texttt{xmm5} en nuestro código) pues
se trata de cuatro valores.

En resumen, entonces, el método es el siguiente: en cada paso levantamos 8 columnas
de datos (en 3 filas por supuesto); desde la columna $ 8*i+1 $ hasta la $ 8*i+8 $.
Numerémoslas desde 1 hasta 8 sin importar si se trata del principio de la fila.
Inmediatamente comprimimos en X esas columnas obteniendo $ (1), (2), ..., (8) $.
Además sabemos que en \texttt{xmm5} tenemos salvados los valores comprimidos de
dos columnas anteriores del paso anterior; las llamamos $ (-1) $ y $ (0) $.
Recordemos que las tenemos comprimidas tanto en X como en Y.

Con esas 10 columnas comprimidas vamos a poder encontrar 8 derivadas en X definitivas
($ [0], [1], ..., [7] $). Estos valores parciales no los volcamos en la imagen destino
como hacíamos en el caso de Prewitt y Sobel, si no que los salvamos en un XMM como
8 enteros de 16 bits (con signo) (\texttt{xmm4} en nuestro código).
Pero además es menester salvar en \texttt{xmm5} los valores
$ (7) $ y $ (8) $, pues son quienes en la próxima iteración actuarán como
$ (-1) $ y $ (0) $ para X. Entonces los salvamos sin perder los valores allí
reservados para Y del paso anterior.

Con Y hacemos lo mismo, y finalmente también salvamos las columnas comprimidas en Y
$ (7) $ y $ (8) $ para el siguiente paso. Una vez que tenemos las derivadas en X y en Y
en sendos registros, debemos saturarlas a 0 para evitar cancelaciones. Esto lo logramos 
con \textbf{\texttt{PMAXSW}} contra un registro de ceros. Luego hacemos la suma y,
como se explicó anteriormente, las instrucciones \textbf{\texttt{PACKUSWB}} y
\textbf{\texttt{MOVQ}} hacen el resto.


