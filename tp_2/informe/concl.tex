\section{Conclusiones}

Con la realización del presente trabajo pudimos apreciar la mejora que se tiene
al implementar un algoritmo con instrucciones del tipo SIMD.
De acuerdo a la experimentación realizada, pudimos observar que esta mejora consiste
en una eficiencia aproximadamente 10 veces más grande.

Si bien los resultados obtenidos son óptimos es evidente la limitación que en 
principio tienen estas instrucciones ya que como se dijo en la introducción, no 
todos los algoritmos son paralelizables al estilo SIMD. Sin embargo y como se
notó en este trabajo, son ideales para el procesamiento de imágenes.
Pareciera que toda transformación que se quiere aplicar a imágenes (o a sonidos)
encontrará en SIMD una herramienta poderosa.

Como ya dijimos, pese a la relativa simplicidad con la que se pueden llevar a
cabo este tipo de algoritmos en esta arquitectura, queremos mencionar nuevamente
que esto requiere un esfuerzo adicional, y en muchos casos trabas importantes.
Además, el código resultante es mucho menos declarativo, así como también mucho
más sensible a cambios.

Por último queremos observar que no hay motivo para conformarse con los registros
\textbf{XMM}. De la misma manera, nuevas arquitecturas podrían proveer no 8 sino
16 o 32 registros, no de 128 bits sino de 256 o 1024. En ese caso se manifestará
la poca reusabilidad de la paralelización, pues las implementaciones hechas para
8 registros de 128 bits claramente no van a aprovechar los nuevos 16 o 32 registros.

A priori podría pensarse en un esquema en el que el código no esté tan atado a
cómo acomodar los datos, sino que tenga sentido en diferentes arquitecturas de
cálculo en paralelo, aprovechando en cada caso la cantidad provista de registros
y bits. Para ello haría falta trabajar con código de más alto nivel, pensando en
qué operaciones se deben paralelizar pero no de a cuántas. Las cuestiones de 
acomodamiento de datos quedaría en manos de un compilador. Es una posibilidad
muy complicada pero también muy interesante, y sólo con el paso del tiempo 
conoceremos su factibilidad.
