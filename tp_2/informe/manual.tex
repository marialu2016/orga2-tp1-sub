\section{Apéndice A: Manual de usuario}

El programa realizado permite aplicar diferentes implementaciones de detección de bordes a imágenes. Tanto el nombre de las imagen fuente como los operadores se indican por línea de comandos de la siguiente manera (suponiendo que el ejecutable está en "exe/bordes"):


   {\small \verb=$ exe/bordes {fuente} {destino} {operadores} =}

    
Donde:

%esto es una lista
    * \texttt{fuente}: es la imagen de origen; puede tener cualquier formato soportado por la función cvLoadImage de OpenCV; los formatos más comunes están soportados;
    
    * \texttt{destino}: es el nombre de las imágenes de salida (generadas por el programa) SIN INCLUIR LA EXTENSIÓN; los archivos generados tendrán ese nombre más un sufijo que indica el operador utilizado, más la extensión; la extensión, el formato y el tamaño son obtenidos de la imagen de entrada;
    
    * \texttt{operadores}: una o más claves separadas por espacio; cada clave representa una implementación particular de detección de bordes; las claves posibles y sus significados son los que siguen:

{\small
\begin{verbatim}    
     CLAVE | OPERADOR  | DIRECCION |     IMPLEMENTACIÓN      |      SUFIJO
    -------+-----------+-----------+-------------------------+---------------------
      r1   |  Roberts  |    XY     |       Ensamblador       |   _asm_roberts
      r2   |  Prewitt  |    XY     |          ""             |   _asm_prewitt
      r3   |   Sobel   |    X      |          ""             |   _asm_sobelX
      r4   |   Sobel   |    Y      |          ""             |   _asm_sobelY
      r5   |   Sobel   |    XY     |          ""             |   _asm_sobelXY
           |           |           |                         |
      cv3  |   Sobel   |    X      |        OpenCV           |    _cv_sobelX
      cv4  |   Sobel   |    Y      |          ""             |    _cv_sobelY
      cv5  |   Sobel   |    XY     |          ""             |    _cv_sobelXY
           |           |           |                         |
      c3   |   Sobel   |    X      |          C              |    _c_sobelX
      c4   |   Sobel   |    Y      |          ""             |    _c_sobelY
      c5   |   Sobel   |    XY     |          ""             |    _c_sobelXY
           |           |           |                         |
     push  |  Roberts  |    X      | Ensamblador usando pila | _asm_roberts(push)
           |           |           |                         |
      byn  |   Escala de grises    |                         |      _byn
\end{verbatim}  }                                        

Por ejemplo, si tenemos el ejecutable en \verb="exe/bordes"= y una imagen en \verb="pics/lena.bmp"=, llamando al programa como

{\small \verb=  $ exe/bordes pics/lena.bmp pics/lena byn r1 r2 r3 r4 r5 cv3 cv4 cv5 c3 c4 c5 push =}

se aplicarán todas las implementaciones disponibles y se generarán los archivos \verb="pics/lena_asm_roberts.bmp", "pics/lena_asm_prewitt.bmp"=, etc... Además el programa mostrará por salida estándar la cantidad aproximada de clocks de procesador insumida por cada implementación.

    